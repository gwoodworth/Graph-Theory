\documentclass[12pt]{article}

\usepackage{amssymb,amsmath,amsthm}
\usepackage[top=1in, bottom=1in, left=1.25in, right=1.25in]{geometry}
\usepackage{fancyhdr}
\usepackage{enumerate}
\usepackage{color}
\usepackage{quiver}
\usepackage{blkarray}
\usepackage{subcaption}
%% Import a package useful for displaying graphics.
\usepackage{graphicx}
\usepackage{quiver}

% Comment the following line to use TeX's default font of Computer Modern.
\usepackage{times,txfonts}

% Adjust line spacing an indentation
\parindent 0pt
\parskip 6pt plus 1pt

\newtheoremstyle{ex215}% name of the style to be used
  {18pt}% measure of space to leave above the theorem. E.g.: 3pt
  {12pt}% measure of space to leave below the theorem. E.g.: 3pt
  {}% name of font to use in the body of the theorem
  {}% measure of space to indent
  {\bfseries}% name of head font
  {:}% punctuation between head and body
  {2ex}% space after theorem head; " " = normal interword space
  {}% Manually specify head
\theoremstyle{ex215} 

\makeatletter
\newtheorem*{lemmacore}{Lemma \@currentlabel}
\newenvironment{lemma}[1]
{\def\@currentlabel{#1}\lemmacore}
{\endlemmacore}

\newtheorem*{corollarycore}{Corollary \@currentlabel}
\newenvironment{corollary}[1]
{\def\@currentlabel{#1}\corollarycore}
{\endcorollarycore}


%%%%%%%%%%%%%%%%%%%%%%%%%%%%%%%%%%%%%%%%%%%%%%%%%%%%%%%%%%%%%%%%%%%%%%%%
%
% Stuff for getting the name/document date/title across the header
\RequirePackage{fancyhdr}
\pagestyle{fancy}
\fancyfoot[C]{\ifnum \value{page} > 1\relax\thepage\fi}
\fancyhead[L]{\ifx\@doclabel\@empty\else\@doclabel\fi}
\fancyhead[C]{\ifx\@docauthor\@empty\else\@docdate\fi}
\fancyhead[R]{\ifx\@docauthor\@empty\@docdate\else\@docauthor\fi}
\headheight 15pt


\def\doclabel#1{\gdef\@doclabel{#1}}
\doclabel{Use {\tt\textbackslash doclabel\{MY LABEL\}}.}
\def\docdate#1{\gdef\@docdate{#1}}
\docdate{Use {\tt\textbackslash docdate\{MY DATE\}}.}
\def\docauthor#1{\gdef\@docauthor{#1}}
%\docauthor{Use {\tt\textbackslash docauthor\{MY NAME\}}.}
\let\@docauthor\@empty

\newcounter{probcount}
\newcounter{subprobcount}
\newlength\probsep
\newlength\pshrinking
\newif\iffirstprob

% The following commands allow for a handy override of labels in an enumeration,
% without having the user keep track of what level the enumeration is happening at.
% To use them, just put \listlabel{....} as the first command after a \begin{enumerate}.
% The '....' is used to define the label.  You can access the name of the current label
% counter with \thelistlabel.  Hence \listlabel{\textbf{\Alph{\thelistlabel}:}} would
% generate lists with bold labels A:, B:, C:, etc.
\newcommand{\listlabel}[1]{\def\thelistlabel{\@enumctr}%
            \expandafter\def\csname label\@enumctr\endcsname{#1}}


\newenvironment{subproblems}%
  {\begin{enumerate}\listlabel{\alph{\thelistlabel})}%
  %% Make the enumerate environment think we are making a list in a list:
  \global \@newlistfalse}%
  {\end{enumerate}}%

\newenvironment{problems}%
  {\ifhmode\unskip\par\fi\setcounter{probcount}{0}\probsep\parskip
  \sbox\@tempboxa{\textbf{9.}}\pshrinking\wd\@tempboxa\advance\pshrinking\labelsep
  \advance\linewidth -\pshrinking
  \advance\@totalleftmargin\pshrinking
  \advance\leftskip\pshrinking}%
  {\ifhmode\unskip \par\fi\advance\leftskip-\pshrinking}%

\newcommand{\problem}{%
  \setcounter{subprobcount}{0}%
  \stepcounter{probcount}%
  \def\@currentlabel{\arabic{probcount}}%
  \ifhmode
    \unskip \par
  \fi
%  \addpenalty{-4000}%
  \iffirstprob\else\addvspace\probsep\fi
  \firstprobfalse
  \hskip -\labelwidth\hskip -\labelsep 
  \hbox to\labelwidth{\hss\textbf{\arabic{probcount}.}}\hskip\labelsep
}%

%% The localhead command puts a label on top of a paragraph -- handy for labels
%% like "Solution:"
\newskip\restoreparskip
\newcommand{\localhead}[1]{\par\smallskip\textbf{#1}\nobreak\\}%
%% The following arcane code was added to make the localhead enviroment get along
%% with the ams theorem enviroment (which is based on the latex list enviroment).
%% previously, the command was {\par\smallskip\textbf{#1}\\}, but the list
%% enviroment would also add a par at the end of this paragraph, which would result
%% in an extra blank line.  Our solution now is to put in our own par, but with
%% none of the parskip glue that adds extra space between paragraphs.
%%  \restoreparskip\parskip\parskip 0pt\par\nobreak\noindent\parskip\restoreparskip}%
\newcommand\solution{\localhead{Solution:}}
\newcommand\subsol{\stepcounter{subprobcount}\localhead{Solution, part \alph{subprobcount}:}}
\newcommand\subsoleg[1]{\stepcounter{subprobcount}\par\textbf{Solution, part \alph{subprobcount}:} #1\\}
\def\solver#1{(Solution by #1)\\\vskip-12pt}


%% The default 'article' class has captions that label figures Figure 1, etc.  This is too 
%% formal for homework sets, so we change \caption so that it just puts the caption text.
%% Here I have just modified the \@makecaption command from the article class.
%% Maybe I should change this in the future to let authors decide whether or not to label a figure.

\long\def\@makecaption#1#2{%
  \vskip\abovecaptionskip
  \sbox\@tempboxa{#2}%
  \ifdim \wd\@tempboxa >\hsize
    #2\par
  \else
    \global \@minipagefalse
    \hb@xt@\hsize{\hfil\box\@tempboxa\hfil}%
  \fi
  \vskip\belowcaptionskip}

%% The default implementation of the proof environment starts a new paragraph at the start of the proof,
%% with the full parskip spacing. This only looks good following a theorem statement 
%% if \parskip is set to zero.  We've got a medium sized parskip, so we overide this 
%% questionable decision.
\renewenvironment{proof}[1][\proofname]{\par
  \pushQED{\qed}%
  \normalfont \topsep6\p@\@plus6\p@\relax
  \trivlist
  \@topsep \topsep
  \item[\hskip\labelsep
        \itshape
    #1\@addpunct{.}]\ignorespaces
}{%
  \popQED\endtrivlist\@endpefalse
}
\makeatother

% Shortcuts for blackboard bold number sets (reals, integers, etc.)
\newcommand{\Reals}{\ensuremath{\mathbb R}}
\newcommand{\Nats}{\ensuremath{\mathbb N}}
\newcommand{\Ints}{\ensuremath{\mathbb Z}}
\newcommand{\Rats}{\ensuremath{\mathbb Q}}
\newcommand{\Cplx}{\ensuremath{\mathbb C}}
%% Some equivalents that some people may prefer.
\let\RR\Reals
\let\NN\Nats
\let\II\Ints
\let\CC\Cplx


% The following commands set up the material that appears
% in the header.
\doclabel{Math 663: HW 7}
\docauthor{Glen Woodworth}
\docdate{19 October 2021}

\begin{document}
\begin{problems}

\problem 2.3.2. Determine which trees have Prufer codes that
\begin{subproblems}
\item contain only one value,
\item contain exactly two values, or
\item have distinct values in all positions. 
\end{subproblems}

\subsol
Stars have only one value in their Prufer code.
\subsol
Double brooms have exactly two values in their Prufer code. 
\subsol
Trees with only two leaves have distinct values in all positions of their Prufer code.

\problem 2.2.7. Use Cayley's Formula to prove that the graph obtained from $K_n$ by deleting an edge has $(n-2)n^{n-3}$ spanning trees.

\solution
Consider the endpoints $u$ and $v$ of the edge that is deleted from $K_n$. Then $d(u)=d(v)=n-2$. Without loss of generality, suppose we delete $v$. The resulting graph is $K_{n-1}$, which has $2^{n-3}$ spanning trees by Cayley's Formula. Adding back $v$ and the $n-2$ edges incident to it yields the original graph. Hence, since there are $n-2$ possibilities for connecting $v$ to the spanning trees in $K_{n-1}$, the original graph has $(n-2)2^{n-3}$ spanning trees.
\qed

\problem 2.2.8. Count the following sets of trees with vertex set $[n]$, giving two proofs for each: one using the Prufer correspondence and one by direct counting arguments
\begin{subproblems}
\item trees that have 2 leaves
\item trees that have n-2 leaves
\end{subproblems}
\subsol
First, direct counting. A tree with exactly 2 leaves on $n$ vertices is $P_n$. Thus, there are $n!$ ways to arrange the vertices. Since reversing the order of the vertices produces the same tree, there are $n!/2$ trees with vertex set $[n]$ that have 2 leaves. Now, the Prufer correspondence counting. Note that all the entries of the Prufer code will be distinct. Then, applying the Prufer code, there are $n$ options for the first entry, then $n-1$ choices for the next one, and $n-3$ for the next, etc. This continues until there are 2 vertices left. Hence, there a $n!/2$ trees that have 2 leaves.
\subsol
First, Prufer correspondence counting. Note that trees that have $n-2$ leaves are double brooms. Thus, there are exactly two values in the Prufer code. There are $n$ options for the first entry, then $n-1$ options for the second entry. For the remaining $n-4$ slots in the Prufer code, each has 2 choices. Hence, there are $n(n-1)2^{n-4}$ trees that have $n-2$ leaves. Now we count directly. There are $n\choose 2$ ways to pick the two centers of the double broom. There are $n-2$ neighboring vertices left to assign. There are 2 choices for each neighbor, either one center or the other. Yet, since flipping the graph over its double center produces the same graph, ${n\choose2 }2^{n-2}$ overcounts by a factor of $2$. Hence, there are ${n\choose2 }2^{n-3}$ trees that have $n-2$ leaves, which equivalent to the result from Prufer correspondence.
\qed

\problem 2.3.2. Prove or disprove: If $T$ is a minimum-weight spanning tree of a weighted graph $G$, then the $u,v$-path in $T$ is a minimum-weight $u,v$-path in $G$.  
\solution
False. Let $G$ be the graph below. $T=\{vw,ws,st,tu\}$ is a minimum weight spanning tree for $G$. Yet, on $T$, $d(u,v)=7$ while on $G$, $d(u,v)=5$.
% https://q.uiver.app/?q=WzAsNSxbMCwxLCJ0IFxcYnVsbGV0Il0sWzEsMCwiXFxidWxsZXQgdSJdLFsyLDEsIlxcYnVsbGV0IHciXSxbMywxLCJcXGJ1bGxldCB2Il0sWzEsMiwiXFxidWxsZXQgcyJdLFswLDQsIjEiLDIseyJzdHlsZSI6eyJoZWFkIjp7Im5hbWUiOiJub25lIn19fV0sWzQsMiwiMSIsMix7InN0eWxlIjp7ImhlYWQiOnsibmFtZSI6Im5vbmUifX19XSxbMiwzLCIzIiwyLHsic3R5bGUiOnsiaGVhZCI6eyJuYW1lIjoibm9uZSJ9fX1dLFsyLDEsIjIiLDIseyJzdHlsZSI6eyJoZWFkIjp7Im5hbWUiOiJub25lIn19fV0sWzAsMSwiMiIsMCx7InN0eWxlIjp7ImhlYWQiOnsibmFtZSI6Im5vbmUifX19XV0=
\[\begin{tikzcd}
	& {\bullet u} \\
	{t \bullet} && {\bullet w} & {\bullet v} \\
	& {\bullet s}
	\arrow["1"', no head, from=2-1, to=3-2]
	\arrow["1"', no head, from=3-2, to=2-3]
	\arrow["3"', no head, from=2-3, to=2-4]
	\arrow["2"', no head, from=2-3, to=1-2]
	\arrow["2", no head, from=2-1, to=1-2]
\end{tikzcd}\]
\qed
\problem
2.3.4. In the graph below, assign weights (1,1,2,2,3,3,4,4) in two ways one way so that the minimum-weight spanning tree is unique, and another way so that the minimum-weight spanning tree is not unique.
\solution
In the left graph below, $T=\{ab,bc,ce,cd\}$ and $T'=\{ab,bc,cd,de\}$ are distinct minimum weight spanning trees. In the right graph below, $T'=\{ab,bc,cd,de\}$ is the only minimum weight spanning tree.
% https://q.uiver.app/?q=WzAsMTEsWzAsMSwiYVxcYnVsbGV0Il0sWzIsMSwiXFxidWxsZXQgYiJdLFswLDMsImQgXFxidWxsZXQiXSxbMiwzLCJcXGJ1bGxldCBjIl0sWzEsMiwiZVxcYnVsbGV0Il0sWzAsMF0sWzMsMSwiYVxcYnVsbGV0Il0sWzUsMSwiXFxidWxsZXQgYiJdLFs1LDMsIlxcYnVsbGV0IGMiXSxbMywzLCJkXFxidWxsZXQiXSxbNCwyLCJlXFxidWxsZXQiXSxbMCwxLCIxIiwwLHsic3R5bGUiOnsiaGVhZCI6eyJuYW1lIjoibm9uZSJ9fX1dLFsxLDMsIjEiLDAseyJzdHlsZSI6eyJoZWFkIjp7Im5hbWUiOiJub25lIn19fV0sWzIsMywiMiIsMix7InN0eWxlIjp7ImhlYWQiOnsibmFtZSI6Im5vbmUifX19XSxbMiwwLCIyIiwwLHsic3R5bGUiOnsiaGVhZCI6eyJuYW1lIjoibm9uZSJ9fX1dLFswLDQsIjQiLDEseyJzdHlsZSI6eyJoZWFkIjp7Im5hbWUiOiJub25lIn19fV0sWzQsMywiMyIsMSx7InN0eWxlIjp7ImhlYWQiOnsibmFtZSI6Im5vbmUifX19XSxbNCwxLCI0IiwxLHsic3R5bGUiOnsiaGVhZCI6eyJuYW1lIjoibm9uZSJ9fX1dLFs0LDIsIjMiLDEseyJzdHlsZSI6eyJoZWFkIjp7Im5hbWUiOiJub25lIn19fV0sWzYsMTAsIjQiLDEseyJzdHlsZSI6eyJoZWFkIjp7Im5hbWUiOiJub25lIn19fV0sWzcsMTAsIjQiLDEseyJzdHlsZSI6eyJoZWFkIjp7Im5hbWUiOiJub25lIn19fV0sWzcsNiwiMSIsMix7InN0eWxlIjp7ImhlYWQiOnsibmFtZSI6Im5vbmUifX19XSxbOSw2LCIzIiwwLHsic3R5bGUiOnsiaGVhZCI6eyJuYW1lIjoibm9uZSJ9fX1dLFs5LDgsIjIiLDIseyJzdHlsZSI6eyJoZWFkIjp7Im5hbWUiOiJub25lIn19fV0sWzgsNywiMSIsMix7InN0eWxlIjp7ImhlYWQiOnsibmFtZSI6Im5vbmUifX19XSxbOCwxMCwiMyIsMSx7InN0eWxlIjp7ImhlYWQiOnsibmFtZSI6Im5vbmUifX19XSxbMTAsOSwiMiIsMSx7InN0eWxlIjp7ImhlYWQiOnsibmFtZSI6Im5vbmUifX19XV0=
\[\begin{tikzcd}
	{} \\
	a\bullet && {\bullet b} & a\bullet && {\bullet b} \\
	& e\bullet &&& e\bullet \\
	{d \bullet} && {\bullet c} & d\bullet && {\bullet c}
	\arrow["1", no head, from=2-1, to=2-3]
	\arrow["1", no head, from=2-3, to=4-3]
	\arrow["2"', no head, from=4-1, to=4-3]
	\arrow["2", no head, from=4-1, to=2-1]
	\arrow["4"{description}, no head, from=2-1, to=3-2]
	\arrow["3"{description}, no head, from=3-2, to=4-3]
	\arrow["4"{description}, no head, from=3-2, to=2-3]
	\arrow["3"{description}, no head, from=3-2, to=4-1]
	\arrow["4"{description}, no head, from=2-4, to=3-5]
	\arrow["4"{description}, no head, from=2-6, to=3-5]
	\arrow["1"', no head, from=2-6, to=2-4]
	\arrow["3", no head, from=4-4, to=2-4]
	\arrow["2"', no head, from=4-4, to=4-6]
	\arrow["1"', no head, from=4-6, to=2-6]
	\arrow["3"{description}, no head, from=4-6, to=3-5]
	\arrow["2"{description}, no head, from=3-5, to=4-4]
\end{tikzcd}\]
\problem 2.3.9. Let $F$ be a spanning forest of a connected weighted graph $G$. Among all edges of $G$ having endpoints in different components of $F$, let $e$ be one of minimum weight. Prove that among all the spanning trees of $G$ that contain $F$, there is one of minimum weight that contains $e$. Use this to give another proof that Kruskal's Algorithm works.
\solution 
Let $T$ be a minimum weight spanning forest containing $F$. Consider two components, $C$ and $C'$ of $F$ connected by $e$ in $G$. If $T$ connects $C$ to $C'$ directly, then it either uses $e$ or another edge with the same weight as $e$. Without loss of generality, suppose in that case that $T$ uses $e$. Then, to connect $C$ to $C'$, $T$ either uses $e$ or there is a path through other components of $F$ not containing $e$ of equal or lesser weight compared to $e$. Hence, a path from $C$ to $C'$ in $T$ that doesn't use $e$ must go through other components using connecting edges that have weights equal to or greater than the weight of $e$. Even if the edges between components of $F$ that $T$ uses are of the same weight as $e$, $T$ would have to use at least two of them to connect $C$ to $C'$ because it is moving through at least one extra component in $F$. Hence, this path will be of more weight than a path containing $e$. Therefore, T is a minimum weight spanning tree containing $F$ that contains $e$. To show that Kruskal's algorithm works, we consider a spanning forest $F$ of $G$ that has no edges. This is the same start as Kruskal's algorithm. Then, we add an edge $e$ that reduces the number of components in $F$ and is of minimum weight. By the above proof, there must be a minimum weight spanning tree containing $e$. This will be true for each step of Kruskal's algorithm by the above proof. Hence, Kruskal's algorithm must be true.
\end{problems}
\end{document}