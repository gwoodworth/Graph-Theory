\documentclass[12pt]{article}

\usepackage{amssymb,amsmath,amsthm}
\usepackage[top=1in, bottom=1in, left=1.25in, right=1.25in]{geometry}
\usepackage{fancyhdr}
\usepackage{enumerate}
\usepackage{color}
\usepackage{quiver}
\usepackage{blkarray}
\usepackage{subcaption}
%% Import a package useful for displaying graphics.
\usepackage{graphicx}
\usepackage{quiver}

% Comment the following line to use TeX's default font of Computer Modern.
\usepackage{times,txfonts}

% Adjust line spacing an indentation
\parindent 0pt
\parskip 6pt plus 1pt

\newtheoremstyle{ex215}% name of the style to be used
  {18pt}% measure of space to leave above the theorem. E.g.: 3pt
  {12pt}% measure of space to leave below the theorem. E.g.: 3pt
  {}% name of font to use in the body of the theorem
  {}% measure of space to indent
  {\bfseries}% name of head font
  {:}% punctuation between head and body
  {2ex}% space after theorem head; " " = normal interword space
  {}% Manually specify head
\theoremstyle{ex215} 

\makeatletter
\newtheorem*{lemmacore}{Lemma \@currentlabel}
\newenvironment{lemma}[1]
{\def\@currentlabel{#1}\lemmacore}
{\endlemmacore}

\newtheorem*{corollarycore}{Corollary \@currentlabel}
\newenvironment{corollary}[1]
{\def\@currentlabel{#1}\corollarycore}
{\endcorollarycore}


%%%%%%%%%%%%%%%%%%%%%%%%%%%%%%%%%%%%%%%%%%%%%%%%%%%%%%%%%%%%%%%%%%%%%%%%
%
% Stuff for getting the name/document date/title across the header
\RequirePackage{fancyhdr}
\pagestyle{fancy}
\fancyfoot[C]{\ifnum \value{page} > 1\relax\thepage\fi}
\fancyhead[L]{\ifx\@doclabel\@empty\else\@doclabel\fi}
\fancyhead[C]{\ifx\@docauthor\@empty\else\@docdate\fi}
\fancyhead[R]{\ifx\@docauthor\@empty\@docdate\else\@docauthor\fi}
\headheight 15pt


\def\doclabel#1{\gdef\@doclabel{#1}}
\doclabel{Use {\tt\textbackslash doclabel\{MY LABEL\}}.}
\def\docdate#1{\gdef\@docdate{#1}}
\docdate{Use {\tt\textbackslash docdate\{MY DATE\}}.}
\def\docauthor#1{\gdef\@docauthor{#1}}
%\docauthor{Use {\tt\textbackslash docauthor\{MY NAME\}}.}
\let\@docauthor\@empty

\newcounter{probcount}
\newcounter{subprobcount}
\newlength\probsep
\newlength\pshrinking
\newif\iffirstprob

% The following commands allow for a handy override of labels in an enumeration,
% without having the user keep track of what level the enumeration is happening at.
% To use them, just put \listlabel{....} as the first command after a \begin{enumerate}.
% The '....' is used to define the label.  You can access the name of the current label
% counter with \thelistlabel.  Hence \listlabel{\textbf{\Alph{\thelistlabel}:}} would
% generate lists with bold labels A:, B:, C:, etc.
\newcommand{\listlabel}[1]{\def\thelistlabel{\@enumctr}%
            \expandafter\def\csname label\@enumctr\endcsname{#1}}


\newenvironment{subproblems}%
  {\begin{enumerate}\listlabel{\alph{\thelistlabel})}%
  %% Make the enumerate environment think we are making a list in a list:
  \global \@newlistfalse}%
  {\end{enumerate}}%

\newenvironment{problems}%
  {\ifhmode\unskip\par\fi\setcounter{probcount}{0}\probsep\parskip
  \sbox\@tempboxa{\textbf{9.}}\pshrinking\wd\@tempboxa\advance\pshrinking\labelsep
  \advance\linewidth -\pshrinking
  \advance\@totalleftmargin\pshrinking
  \advance\leftskip\pshrinking}%
  {\ifhmode\unskip \par\fi\advance\leftskip-\pshrinking}%

\newcommand{\problem}{%
  \setcounter{subprobcount}{0}%
  \stepcounter{probcount}%
  \def\@currentlabel{\arabic{probcount}}%
  \ifhmode
    \unskip \par
  \fi
%  \addpenalty{-4000}%
  \iffirstprob\else\addvspace\probsep\fi
  \firstprobfalse
  \hskip -\labelwidth\hskip -\labelsep 
  \hbox to\labelwidth{\hss\textbf{\arabic{probcount}.}}\hskip\labelsep
}%

%% The localhead command puts a label on top of a paragraph -- handy for labels
%% like "Solution:"
\newskip\restoreparskip
\newcommand{\localhead}[1]{\par\smallskip\textbf{#1}\nobreak\\}%
%% The following arcane code was added to make the localhead enviroment get along
%% with the ams theorem enviroment (which is based on the latex list enviroment).
%% previously, the command was {\par\smallskip\textbf{#1}\\}, but the list
%% enviroment would also add a par at the end of this paragraph, which would result
%% in an extra blank line.  Our solution now is to put in our own par, but with
%% none of the parskip glue that adds extra space between paragraphs.
%%  \restoreparskip\parskip\parskip 0pt\par\nobreak\noindent\parskip\restoreparskip}%
\newcommand\solution{\localhead{Solution:}}
\newcommand\subsol{\stepcounter{subprobcount}\localhead{Solution, part \alph{subprobcount}:}}
\newcommand\subsoleg[1]{\stepcounter{subprobcount}\par\textbf{Solution, part \alph{subprobcount}:} #1\\}
\def\solver#1{(Solution by #1)\\\vskip-12pt}


%% The default 'article' class has captions that label figures Figure 1, etc.  This is too 
%% formal for homework sets, so we change \caption so that it just puts the caption text.
%% Here I have just modified the \@makecaption command from the article class.
%% Maybe I should change this in the future to let authors decide whether or not to label a figure.

\long\def\@makecaption#1#2{%
  \vskip\abovecaptionskip
  \sbox\@tempboxa{#2}%
  \ifdim \wd\@tempboxa >\hsize
    #2\par
  \else
    \global \@minipagefalse
    \hb@xt@\hsize{\hfil\box\@tempboxa\hfil}%
  \fi
  \vskip\belowcaptionskip}

%% The default implementation of the proof environment starts a new paragraph at the start of the proof,
%% with the full parskip spacing. This only looks good following a theorem statement 
%% if \parskip is set to zero.  We've got a medium sized parskip, so we overide this 
%% questionable decision.
\renewenvironment{proof}[1][\proofname]{\par
  \pushQED{\qed}%
  \normalfont \topsep6\p@\@plus6\p@\relax
  \trivlist
  \@topsep \topsep
  \item[\hskip\labelsep
        \itshape
    #1\@addpunct{.}]\ignorespaces
}{%
  \popQED\endtrivlist\@endpefalse
}
\makeatother

% Shortcuts for blackboard bold number sets (reals, integers, etc.)
\newcommand{\Reals}{\ensuremath{\mathbb R}}
\newcommand{\Nats}{\ensuremath{\mathbb N}}
\newcommand{\Ints}{\ensuremath{\mathbb Z}}
\newcommand{\Rats}{\ensuremath{\mathbb Q}}
\newcommand{\Cplx}{\ensuremath{\mathbb C}}
%% Some equivalents that some people may prefer.
\let\RR\Reals
\let\NN\Nats
\let\II\Ints
\let\CC\Cplx


% The following commands set up the material that appears
% in the header.
\doclabel{Math 663: Midterm 1 Re-write}
\docauthor{Glen Woodworth}
\docdate{11 October 2021}

\begin{document}
\begin{problems}

\problem  Let $G$ be a simple graph such that $\delta(G)=k \geq 2.$ Prove that $G$ contains a cycle on at least $k+1$ vertices. (3 pts extra credit: Give an example that demonstrates this result is best possible.)

\solution
Let $G$ be a simple graph such that $\delta(G)=k \geq 2.$ Let $u$ be the endpoint of a maximal length path in $G$. Since $P$ is maximal, every neighbor of $u$ is in $V(P)$. Hence, since $u$ has at least $k$ neighbors, $P$ has at least $k$ edges. Let $v$ be the farthest neighbor of $u$'s along $P$. Then there is a $uv$-path that has length $k$, but there is also edge $uv$. Hence, there is a cycle of length $k+1$. Consider the graph below, which has minimum degree 2 with $C_3$ as the maximum length cycle, as a demonstration that this result is best possible:
% https://q.uiver.app/?q=WzAsMyxbMCwwLCJcXGJ1bGxldCJdLFsxLDAsIlxcYnVsbGV0Il0sWzIsMCwiXFxidWxsZXQiXSxbMCwxLCIiLDEseyJzdHlsZSI6eyJoZWFkIjp7Im5hbWUiOiJub25lIn19fV0sWzEsMiwiIiwxLHsic3R5bGUiOnsiaGVhZCI6eyJuYW1lIjoibm9uZSJ9fX1dLFsyLDAsIiIsMSx7ImN1cnZlIjoyLCJzdHlsZSI6eyJoZWFkIjp7Im5hbWUiOiJub25lIn19fV1d
\[\begin{tikzcd}
	\bullet & \bullet & \bullet
	\arrow[no head, from=1-1, to=1-2]
	\arrow[no head, from=1-2, to=1-3]
	\arrow[curve={height=12pt}, no head, from=1-3, to=1-1]
\end{tikzcd}\]
\qed
\problem Let $G$ be a connected graph on $n$ vertices. Prove that $G$ has exactly one cycle if and only if $G$ has exactly $n$ edges.
\solution
$\rightarrow$\\Suppose $G$ has exactly one cycle and $G$ is connected on $n$ vertices. Let $C_k$ be the cycle, $k\le n$. If $k=n$, then $G$ has $n$ edges, as desired. On the other hand, suppose $k<n$. Consider a graph containing only $C_k$. To yield $G$, we must connect $n-k$ vertices to $C_k$. Yet, each edge added must be a cut-edge in order to avoid creating a cycle. Thus, exactly one edge must be added for each of the $n-k$ missing vertices. Hence, the total number of edges in $G$ is:
 \begin{align*}
e(G)&=e(C_k)+n-k\\
&=k+n-k\\
&=n
\end{align*}
$\leftarrow$\\Suppose $G$ is a connected graph on $n$ vertices and $n$ edges. $G$ is not a tree because $G$ has one too many edges to be a tree. Since $G$ is connected but not a tree, $G$ contains a cycle. Remove an edge from cycle $C_k$ in $G$ to produce $G'$ with $n$ vertices and $n-1$ edges. Since the edge was in a cycle, it is not a cut-edge. Thus, $G'$ is connected on $n-1$ edges, which means $G'$ is a tree. Therefore, $G'$ is acyclic. Hence, $C_k$ is the only cycle in $G$.
\qed
\problem Let $G$ be a simple graph having no isolated vertex and no \emph{induced} subgraph with exactly two edges. Prove that $G$ must be the complete graph.
\solution
Assume $G$ is a simple graph with no isolated vertices, and no induced subgraph with exactly two edges. Assume, for a contradiction, that $G$ is not complete. That is, there exists $u,v\in V(G)$ such that $uv\notin E(G)$. Since there are no isolated vertices in $G$, $u$ and $v$ have neighbors. Delete every vertex in $G$ except for $u$, $v$, a neighbor of $u$, and a neighbor of $v$. If $u$ and $v$ share one or more neighbors, delete every vertex except $u$, $v$, and one vertex from their neighborhood. In either case, the induced subgraph has exactly two edges, a contradiction. Hence, $G$ must be the complete graph.
\qed

\problem Let $d$ be a sequence $d_1,d_2,\cdots,d_n$ be positive integers and assume $n \geq 2.$ Prove that there exists a tree $T$ with degree sequence $d$ if and only if $\sum_{i=1}^n d_i = 2n-2.$
\solution
$\rightarrow$\\
Suppose there exists tree $T$ with degree sequence $d_1,d_2,...,d_n$, $n\ge 2$. Then $T$ has $n-1$ edges. Hence, the degree sum of $T$ is $\sum\limits_{i=1}^{n}d_i=2n-2$, as desired.\\
$\leftarrow$\\
Let $d$ be a sequence $d_1,d_2,\cdots,d_n$ of positive integers and, without loss of generality, suppose they are arranged in non-increasing order. Assume $n \geq 2.$ Suppose $\sum\limits_{i=1}^{n}d_i=2n-2$. I claim $d_n=d_{n-1}=1$. Otherwise, if either $d_{n-1}\ge 2$ and $d_n=1$, or if $d_{n-1},d_n\ge 2$, then $\sum\limits_{i=1}^{n}d_i\ge 2n-1$, which is a contradiction. Let $\Delta =max_{1\le i\le n}d_i$. The maximum $\Delta$ could be is when all the other terms of $d$ are equal to 1. In this extremal case, we have:
\begin{align*}
\sum\limits_{i=1}^{n}d_i&=2n-2=\Delta+n-1.
\end{align*}
Which means,
\[
n-1=\Delta
\]
in this extremal case. Hence, in general, $\Delta\le n-1$. Furthermore, for $n>2$, $\Delta\ge 2$, for otherwise, $\sum\limits_{i=1}^{n}d_i=n-2$, a contradiction. Proceeding inductively on $n$, when $n=2$, $d=1,1$, which is certainly the degree sequence of a tree. Let $2< k\in\Nats$. For our inductive hypothesis, suppose that for a non-increasing sequence $d=d_1,d_2,\cdots,d_{k}$ of positive integers where $\sum\limits_{i=1}^{k}d_i=2k-2$ there exists a tree with degree sequence $d$. Consider the sequence of positive integers arranged in non-increasing order $d' = d_1',d_2',\cdots,d_{k+1}'$ where $\sum\limits_{i=1}^{k+1}d_i'=2(k+1)-2$. Now, remove $d_{k+1}'$ from the sequence and subtract 1 from $d_s'\ge 2$, $1\le s< k$, ($d_s'$ must exists since $\Delta\ge 2$) to create $d''$. Evaluating the sum of the terms in $d''$, we have:
\begin{align*}
\sum\limits_{i=1}^{k}d_i''&=2(k+1)-2-2\\
&=2k+2-4\\
&=2k-2.
\end{align*}
Hence, by the inductive hypothesis, there exists a tree $T$ with degree sequence $d''$. In order to get a tree with degree sequence $d'$, we simply add a vertex $v$ to $T$ and an edge from $v$ to a vertex with degree $d_s'-1$ in $T$. Since this is merely adding a leaf to a tree, the resulting graph must also be a tree.
\qed
\end{problems}
\end{document}