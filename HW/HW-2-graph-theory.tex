\documentclass[12pt]{article}

\usepackage{amssymb,amsmath,amsthm}
\usepackage[top=1in, bottom=1in, left=1.25in, right=1.25in]{geometry}
\usepackage{fancyhdr}
\usepackage{enumerate}
\usepackage{color}
\usepackage{quiver}
\usepackage{blkarray}
\usepackage{subcaption}
%% Import a package useful for displaying graphics.
\usepackage{graphicx}
\usepackage{quiver}

% Comment the following line to use TeX's default font of Computer Modern.
\usepackage{times,txfonts}

% Adjust line spacing an indentation
\parindent 0pt
\parskip 6pt plus 1pt

\newtheoremstyle{ex215}% name of the style to be used
  {18pt}% measure of space to leave above the theorem. E.g.: 3pt
  {12pt}% measure of space to leave below the theorem. E.g.: 3pt
  {}% name of font to use in the body of the theorem
  {}% measure of space to indent
  {\bfseries}% name of head font
  {:}% punctuation between head and body
  {2ex}% space after theorem head; " " = normal interword space
  {}% Manually specify head
\theoremstyle{ex215} 

\makeatletter
\newtheorem*{lemmacore}{Lemma \@currentlabel}
\newenvironment{lemma}[1]
{\def\@currentlabel{#1}\lemmacore}
{\endlemmacore}

\newtheorem*{corollarycore}{Corollary \@currentlabel}
\newenvironment{corollary}[1]
{\def\@currentlabel{#1}\corollarycore}
{\endcorollarycore}


%%%%%%%%%%%%%%%%%%%%%%%%%%%%%%%%%%%%%%%%%%%%%%%%%%%%%%%%%%%%%%%%%%%%%%%%
%
% Stuff for getting the name/document date/title across the header
\RequirePackage{fancyhdr}
\pagestyle{fancy}
\fancyfoot[C]{\ifnum \value{page} > 1\relax\thepage\fi}
\fancyhead[L]{\ifx\@doclabel\@empty\else\@doclabel\fi}
\fancyhead[C]{\ifx\@docauthor\@empty\else\@docdate\fi}
\fancyhead[R]{\ifx\@docauthor\@empty\@docdate\else\@docauthor\fi}
\headheight 15pt


\def\doclabel#1{\gdef\@doclabel{#1}}
\doclabel{Use {\tt\textbackslash doclabel\{MY LABEL\}}.}
\def\docdate#1{\gdef\@docdate{#1}}
\docdate{Use {\tt\textbackslash docdate\{MY DATE\}}.}
\def\docauthor#1{\gdef\@docauthor{#1}}
%\docauthor{Use {\tt\textbackslash docauthor\{MY NAME\}}.}
\let\@docauthor\@empty

\newcounter{probcount}
\newcounter{subprobcount}
\newlength\probsep
\newlength\pshrinking
\newif\iffirstprob

% The following commands allow for a handy override of labels in an enumeration,
% without having the user keep track of what level the enumeration is happening at.
% To use them, just put \listlabel{....} as the first command after a \begin{enumerate}.
% The '....' is used to define the label.  You can access the name of the current label
% counter with \thelistlabel.  Hence \listlabel{\textbf{\Alph{\thelistlabel}:}} would
% generate lists with bold labels A:, B:, C:, etc.
\newcommand{\listlabel}[1]{\def\thelistlabel{\@enumctr}%
            \expandafter\def\csname label\@enumctr\endcsname{#1}}


\newenvironment{subproblems}%
  {\begin{enumerate}\listlabel{\alph{\thelistlabel})}%
  %% Make the enumerate environment think we are making a list in a list:
  \global \@newlistfalse}%
  {\end{enumerate}}%

\newenvironment{problems}%
  {\ifhmode\unskip\par\fi\setcounter{probcount}{0}\probsep\parskip
  \sbox\@tempboxa{\textbf{9.}}\pshrinking\wd\@tempboxa\advance\pshrinking\labelsep
  \advance\linewidth -\pshrinking
  \advance\@totalleftmargin\pshrinking
  \advance\leftskip\pshrinking}%
  {\ifhmode\unskip \par\fi\advance\leftskip-\pshrinking}%

\newcommand{\problem}{%
  \setcounter{subprobcount}{0}%
  \stepcounter{probcount}%
  \def\@currentlabel{\arabic{probcount}}%
  \ifhmode
    \unskip \par
  \fi
%  \addpenalty{-4000}%
  \iffirstprob\else\addvspace\probsep\fi
  \firstprobfalse
  \hskip -\labelwidth\hskip -\labelsep 
  \hbox to\labelwidth{\hss\textbf{\arabic{probcount}.}}\hskip\labelsep
}%

%% The localhead command puts a label on top of a paragraph -- handy for labels
%% like "Solution:"
\newskip\restoreparskip
\newcommand{\localhead}[1]{\par\smallskip\textbf{#1}\nobreak\\}%
%% The following arcane code was added to make the localhead enviroment get along
%% with the ams theorem enviroment (which is based on the latex list enviroment).
%% previously, the command was {\par\smallskip\textbf{#1}\\}, but the list
%% enviroment would also add a par at the end of this paragraph, which would result
%% in an extra blank line.  Our solution now is to put in our own par, but with
%% none of the parskip glue that adds extra space between paragraphs.
%%  \restoreparskip\parskip\parskip 0pt\par\nobreak\noindent\parskip\restoreparskip}%
\newcommand\solution{\localhead{Solution:}}
\newcommand\subsol{\stepcounter{subprobcount}\localhead{Solution, part \alph{subprobcount}:}}
\newcommand\subsoleg[1]{\stepcounter{subprobcount}\par\textbf{Solution, part \alph{subprobcount}:} #1\\}
\def\solver#1{(Solution by #1)\\\vskip-12pt}


%% The default 'article' class has captions that label figures Figure 1, etc.  This is too 
%% formal for homework sets, so we change \caption so that it just puts the caption text.
%% Here I have just modified the \@makecaption command from the article class.
%% Maybe I should change this in the future to let authors decide whether or not to label a figure.

\long\def\@makecaption#1#2{%
  \vskip\abovecaptionskip
  \sbox\@tempboxa{#2}%
  \ifdim \wd\@tempboxa >\hsize
    #2\par
  \else
    \global \@minipagefalse
    \hb@xt@\hsize{\hfil\box\@tempboxa\hfil}%
  \fi
  \vskip\belowcaptionskip}

%% The default implementation of the proof environment starts a new paragraph at the start of the proof,
%% with the full parskip spacing. This only looks good following a theorem statement 
%% if \parskip is set to zero.  We've got a medium sized parskip, so we overide this 
%% questionable decision.
\renewenvironment{proof}[1][\proofname]{\par
  \pushQED{\qed}%
  \normalfont \topsep6\p@\@plus6\p@\relax
  \trivlist
  \@topsep \topsep
  \item[\hskip\labelsep
        \itshape
    #1\@addpunct{.}]\ignorespaces
}{%
  \popQED\endtrivlist\@endpefalse
}
\makeatother

% Shortcuts for blackboard bold number sets (reals, integers, etc.)
\newcommand{\Reals}{\ensuremath{\mathbb R}}
\newcommand{\Nats}{\ensuremath{\mathbb N}}
\newcommand{\Ints}{\ensuremath{\mathbb Z}}
\newcommand{\Rats}{\ensuremath{\mathbb Q}}
\newcommand{\Cplx}{\ensuremath{\mathbb C}}
%% Some equivalents that some people may prefer.
\let\RR\Reals
\let\NN\Nats
\let\II\Ints
\let\CC\Cplx


% The following commands set up the material that appears
% in the header.
\doclabel{Math 663: Homework 3}
\docauthor{Glen Woodworth}
\docdate{14 September 2021}

\begin{document}
\begin{problems}
\problem 1.3.4 Corollary. In a graph $G$ the average vertex degree is $\frac{2e(G)}{n(G)}$, and hence $\delta(G)\le\frac{2e(G)}{n(G)}\le\Delta(G)$.

\solution 
\qed

\problem 1.3.5 Corollary. Every graph has an even number of vertices of odd degree. No graph of odd order is regular with odd degree.
\solution 
We proceed by proof by contradiction. Suppose $G$ has an odd number of vertices of odd degree. We can split the sum of the degrees of the vertices of $G$ into even and odd. Let $j,k,m,n\in\Nats$ Observe:
\begin{align*}
\sum\limits_{v\in V(G)}d(v)&=2k_1+2k_2+1+...+2_k_n+1+2j_1+2j_2+...+2j_m\\
\end{align*}\qed

\problem 1.2.8: Determine the values $m$ and $n$ such that $K_{m,n}$ is Eulerian.

\solution
From Thm.1.2.6, a graph $G$ is Eulerian if and only if it has at most one nontrivial component and its vertices all have even degree. Thus, $m=n\ge 2x$ for $x\;\epsilon\; \Nats$. \qed
\pagebreak
\problem 1.2.11: Prove or disprove: If $G$ is an Eulerian graph with edges $e$, $f$ that share a vertex, then $G$ has an Eulerian circuit in which $e$, $f$ appear consecutively.

\solution
False. See graph below.
% https://q.uiver.app/?q=WzAsNyxbMSwxLCJcXGJ1bGxldCJdLFsyLDEsIlxcYnVsbGV0Il0sWzEsMiwiXFxidWxsZXQiXSxbMiwyLCJcXGJ1bGxldCJdLFsxLDAsIlxcYnVsbGV0Il0sWzAsMSwiXFxidWxsZXQiXSxbMywxXSxbNSwwLCJmIiwxLHsic3R5bGUiOnsiaGVhZCI6eyJuYW1lIjoibm9uZSJ9fX1dLFswLDIsIiIsMSx7InN0eWxlIjp7ImhlYWQiOnsibmFtZSI6Im5vbmUifX19XSxbMiwzLCIiLDEseyJzdHlsZSI6eyJoZWFkIjp7Im5hbWUiOiJub25lIn19fV0sWzMsMSwiIiwxLHsic3R5bGUiOnsiaGVhZCI6eyJuYW1lIjoibm9uZSJ9fX1dLFswLDQsImUiLDEseyJzdHlsZSI6eyJoZWFkIjp7Im5hbWUiOiJub25lIn19fV0sWzEsMCwiIiwxLHsic3R5bGUiOnsiaGVhZCI6eyJuYW1lIjoibm9uZSJ9fX1dLFs0LDUsIiIsMSx7InN0eWxlIjp7ImhlYWQiOnsibmFtZSI6Im5vbmUifX19XV0=
\[\begin{tikzcd}
	& \bullet \\
	\bullet & \bullet & \bullet & {} \\
	& \bullet & \bullet
	\arrow["f"{description}, no head, from=2-1, to=2-2]
	\arrow[no head, from=2-2, to=3-2]
	\arrow[no head, from=3-2, to=3-3]
	\arrow[no head, from=3-3, to=2-3]
	\arrow["e"{description}, no head, from=2-2, to=1-2]
	\arrow[no head, from=2-3, to=2-2]
	\arrow[no head, from=1-2, to=2-1]
\end{tikzcd}\]
\qed
\problem 1.2.22: Prove that a graph is connected if and only if for every partition of its vertices into two nonempty sets, there is an edge with endpoints in both sets.
\subsol
$\rightarrow$
Suppose $G$ is connected. Being connected means that there must be a path between any two vertices. Therefore, for any nonempty bipartition of the vertices of a connected graph, there will be a path from one partition to the other. Thus, there must be an edge with endpoints in both sets to make the path possible.
\subsol
$\leftarrow$
Suppose for every partition of $V(G)$ into two nonempty sets, there is an edge with endpoints in both sets. Consider a bipartition of $V(G)=b_1\cup b_1'$, such that one set, $b_1$ has only one vertex, $u$, in it while the other has the rest, $b_1'=V(G)$\textbackslash $u$. Thus, there must be an edge from $u_1$ to some vertex in $b_1'$. Call this vertex $u_2$, then create a new partition with call it $b_2=[u_1,u_2]$ and $b_2'=V(G)$\textbackslash $u_1,u_2$. There must also be an edge with endpoints in $b_2$ and $b_2'$. Suppose $V(G)$ has $n\;\epsilon\; \Nats$ vertices. Then, continue this process until $b_{n-1}=[u_1,u_2,...u_{n-1}]$ and $b_{n-1}'=[u_n]$. Thus, by the construction of the $n-1$ different bipartitions, and since there must be an edge from $b_{n-1}$ to $b_n$, there must be $u_1,\; u_n$-walk. Therefore, there is a $u_1,\; u_n$-path. Since $u_1$ is arbitrary, this construction follows for any $u_i,u_j$, $1\le i,j\le n$. Thus, for any $u,v\;\epsilon\; V(G)$, there is a $u,v$-path, as desired.\qed
\problem 1.2.25: Use ordinary induction on the number of edges to prove that absence of odd cycles is a sufficient condition for a graph to be bipartite.
\solution
First, the base case: Consider a graph with $0$ edges. It certainly has no odd cycle. Any vertices it has are all independent. Therefore, $G$ is bipartite.

Now the inductive step. Let $k\ge 0$. Assume for all $G$ with $k$ edges and no odd cycles, $G$ is bipartite. Suppose $G'$ is a graph with length $k+1$ and no odd cycles. Let $xy\;\epsilon\; E(G')$. Consider the graph $G=G'-xy$. By the inductive hypothesis, $G$ is bipartite. Suppose $xy$ is a cut-edge for $G'$. Then $x$ and $y$ are in different components of $G$. Thus, if we need to, we can recolor the components $x$ and $y$ are in, respectively, so that they have different colors without changing any of the other bipartite components of $G$. Therefore, $G+xy=G'$ is bipartite. Now, suppose $xy$ is not a cut edge for $G'$. Then $xy$ lies on a cycle. If $x$ and $y$ are in disjoint independent sets, then we are done, $G'$ is bipartite. Suppose, on the other hand, that $xy$ lie in one independent set of $G'$. In order to make a cycle going through $xy$, there would need to be an even number of edges to go to another independent set and back, then plus one to go from $x$ to $y$. So the cycle $xy$ lies on in this case is odd. But this is a contradiction because we assumed $G'$ contains no odd cycle. Thus, $G'$ is bipartite. By induction, the result follows.  \qed

\problem 1.2.26: Prove that a graph $G$ is bipartite if and only if every subgraph $H$ of $G$ has an independent set consisting of at least half of $V(H)$.
\subsol
$\rightarrow$
Suppose $G$ is bipartite. Let the partite sets of $V(G)$ be $B_1$ and $B_2$. Then $V(H)$ has vertices in $B_1$ or $B_2$. Then the result follows because $B_1$ and $B_2$ are disjoint independent sets.
\subsol
$\rightarrow$
We prove the contrapositive: If $G$ is not bipartite, then there exists a subgraph $H$ of $G$ such that there is no independent set consisting of at least half of $V(H)$.

Suppose $G$ is not bipartite. Then $G$ contains an odd cycle. Let $H$ be an odd cycle in $G$. In order to make the maximal independent set of $H$, we alternate the color of the vertices going around the cycle. But since $H$ is an odd cycle, it has $2k+1$ vertices for $k\; \epsilon\; \Nats$, the maximum vertices in one independent set is half of this. Since it must be an integer, the maximum vertices in one independent set of $H$ is $k$. Since $k<\frac{2k+1}{2}$ and since $k$ is the maximum vertices in one independent set of $H$, we have shown the contrapositive true, so the result follows.

\qed

\problem 1.2.29: Let $G$ be a connected simple graph not having $P_4$ or $C_3$ as an induced subgraph. Prove that $G$ is a biclique (complete bipartite graph).

\solution
Notice that, since $P_4$ cannot be an induced subgraph of $G$, $G$ must not have any $C_k$ or $P_k$ $k>4$. Thus, since $G$ also cannot contain $C_3$, $G$ contains no odd cycle and is therefore bipartite. Let $B_1$ and $B_2$ be the partite sets of $G$. Suppose for some $x\;\epsilon\; B_1$ and some $y\;\epsilon\; B_2$, $x$ is not adjacent to $y$. Since $G$ is connected, there must be some $x\; y$-path but it would have to go from $B_1$ to $B_2$ more than once because we assumed $x$ is not adjacent to $y$. There are no paths $P_4$ or longer, so this is impossible. Thus, $x$ and $y$ are neighbors, and $G$ is a biclique, as desired.
\qed
\end{problems}
\end{document}