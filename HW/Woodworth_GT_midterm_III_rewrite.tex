\documentclass[12pt]{article}

\usepackage{amssymb,amsmath,amsthm}
\usepackage[top=1in, bottom=1in, left=1.25in, right=1.25in]{geometry}
\usepackage{fancyhdr}
\usepackage{enumerate}
\usepackage{color}
\usepackage{quiver}
\usepackage{blkarray}
\usepackage{subcaption}
%% Import a package useful for displaying graphics.
\usepackage{graphicx}
\usepackage{quiver}

% Comment the following line to use TeX's default font of Computer Modern.
\usepackage{times,txfonts}

% Adjust line spacing an indentation
\parindent 0pt
\parskip 6pt plus 1pt

\newtheoremstyle{ex215}% name of the style to be used
  {18pt}% measure of space to leave above the theorem. E.g.: 3pt
  {12pt}% measure of space to leave below the theorem. E.g.: 3pt
  {}% name of font to use in the body of the theorem
  {}% measure of space to indent
  {\bfseries}% name of head font
  {:}% punctuation between head and body
  {2ex}% space after theorem head; " " = normal interword space
  {}% Manually specify head
\theoremstyle{ex215} 

\makeatletter
\newtheorem*{lemmacore}{Lemma \@currentlabel}
\newenvironment{lemma}[1]
{\def\@currentlabel{#1}\lemmacore}
{\endlemmacore}

\newtheorem*{corollarycore}{Corollary \@currentlabel}
\newenvironment{corollary}[1]
{\def\@currentlabel{#1}\corollarycore}
{\endcorollarycore}


%%%%%%%%%%%%%%%%%%%%%%%%%%%%%%%%%%%%%%%%%%%%%%%%%%%%%%%%%%%%%%%%%%%%%%%%
%
% Stuff for getting the name/document date/title across the header
\RequirePackage{fancyhdr}
\pagestyle{fancy}
\fancyfoot[C]{\ifnum \value{page} > 1\relax\thepage\fi}
\fancyhead[L]{\ifx\@doclabel\@empty\else\@doclabel\fi}
\fancyhead[C]{\ifx\@docauthor\@empty\else\@docdate\fi}
\fancyhead[R]{\ifx\@docauthor\@empty\@docdate\else\@docauthor\fi}
\headheight 15pt


\def\doclabel#1{\gdef\@doclabel{#1}}
\doclabel{Use {\tt\textbackslash doclabel\{MY LABEL\}}.}
\def\docdate#1{\gdef\@docdate{#1}}
\docdate{Use {\tt\textbackslash docdate\{MY DATE\}}.}
\def\docauthor#1{\gdef\@docauthor{#1}}
%\docauthor{Use {\tt\textbackslash docauthor\{MY NAME\}}.}
\let\@docauthor\@empty

\newcounter{probcount}
\newcounter{subprobcount}
\newlength\probsep
\newlength\pshrinking
\newif\iffirstprob

% The following commands allow for a handy override of labels in an enumeration,
% without having the user keep track of what level the enumeration is happening at.
% To use them, just put \listlabel{....} as the first command after a \begin{enumerate}.
% The '....' is used to define the label.  You can access the name of the current label
% counter with \thelistlabel.  Hence \listlabel{\textbf{\Alph{\thelistlabel}:}} would
% generate lists with bold labels A:, B:, C:, etc.
\newcommand{\listlabel}[1]{\def\thelistlabel{\@enumctr}%
            \expandafter\def\csname label\@enumctr\endcsname{#1}}


\newenvironment{subproblems}%
  {\begin{enumerate}\listlabel{\alph{\thelistlabel})}%
  %% Make the enumerate environment think we are making a list in a list:
  \global \@newlistfalse}%
  {\end{enumerate}}%

\newenvironment{problems}%
  {\ifhmode\unskip\par\fi\setcounter{probcount}{0}\probsep\parskip
  \sbox\@tempboxa{\textbf{9.}}\pshrinking\wd\@tempboxa\advance\pshrinking\labelsep
  \advance\linewidth -\pshrinking
  \advance\@totalleftmargin\pshrinking
  \advance\leftskip\pshrinking}%
  {\ifhmode\unskip \par\fi\advance\leftskip-\pshrinking}%

\newcommand{\problem}{%
  \setcounter{subprobcount}{0}%
  \stepcounter{probcount}%
  \def\@currentlabel{\arabic{probcount}}%
  \ifhmode
    \unskip \par
  \fi
%  \addpenalty{-4000}%
  \iffirstprob\else\addvspace\probsep\fi
  \firstprobfalse
  \hskip -\labelwidth\hskip -\labelsep 
  \hbox to\labelwidth{\hss\textbf{\arabic{probcount}.}}\hskip\labelsep
}%

%% The localhead command puts a label on top of a paragraph -- handy for labels
%% like "Solution:"
\newskip\restoreparskip
\newcommand{\localhead}[1]{\par\smallskip\textbf{#1}\nobreak\\}%
%% The following arcane code was added to make the localhead enviroment get along
%% with the ams theorem enviroment (which is based on the latex list enviroment).
%% previously, the command was {\par\smallskip\textbf{#1}\\}, but the list
%% enviroment would also add a par at the end of this paragraph, which would result
%% in an extra blank line.  Our solution now is to put in our own par, but with
%% none of the parskip glue that adds extra space between paragraphs.
%%  \restoreparskip\parskip\parskip 0pt\par\nobreak\noindent\parskip\restoreparskip}%
\newcommand\solution{\localhead{Solution:}}
\newcommand\subsol{\stepcounter{subprobcount}\localhead{Solution, part \alph{subprobcount}:}}
\newcommand\subsoleg[1]{\stepcounter{subprobcount}\par\textbf{Solution, part \alph{subprobcount}:} #1\\}
\def\solver#1{(Solution by #1)\\\vskip-12pt}


%% The default 'article' class has captions that label figures Figure 1, etc.  This is too 
%% formal for homework sets, so we change \caption so that it just puts the caption text.
%% Here I have just modified the \@makecaption command from the article class.
%% Maybe I should change this in the future to let authors decide whether or not to label a figure.

\long\def\@makecaption#1#2{%
  \vskip\abovecaptionskip
  \sbox\@tempboxa{#2}%
  \ifdim \wd\@tempboxa >\hsize
    #2\par
  \else
    \global \@minipagefalse
    \hb@xt@\hsize{\hfil\box\@tempboxa\hfil}%
  \fi
  \vskip\belowcaptionskip}

%% The default implementation of the proof environment starts a new paragraph at the start of the proof,
%% with the full parskip spacing. This only looks good following a theorem statement 
%% if \parskip is set to zero.  We've got a medium sized parskip, so we overide this 
%% questionable decision.
\renewenvironment{proof}[1][\proofname]{\par
  \pushQED{\qed}%
  \normalfont \topsep6\p@\@plus6\p@\relax
  \trivlist
  \@topsep \topsep
  \item[\hskip\labelsep
        \itshape
    #1\@addpunct{.}]\ignorespaces
}{%
  \popQED\endtrivlist\@endpefalse
}
\makeatother

% Shortcuts for blackboard bold number sets (reals, integers, etc.)
\newcommand{\Reals}{\ensuremath{\mathbb R}}
\newcommand{\Nats}{\ensuremath{\mathbb N}}
\newcommand{\Ints}{\ensuremath{\mathbb Z}}
\newcommand{\Rats}{\ensuremath{\mathbb Q}}
\newcommand{\Cplx}{\ensuremath{\mathbb C}}
%% Some equivalents that some people may prefer.
\let\RR\Reals
\let\NN\Nats
\let\II\Ints
\let\CC\Cplx


% The following commands set up the material that appears
% in the header.
\doclabel{Math 663: Midterm III Rewrite}
\docauthor{Glen Woodworth}
\docdate{29 November 2021}

\begin{document}
\begin{problems}
\problem Five jobs, five applicants, etc... Complete two iterations of the Gale-Shapley Proposal Algorithm. For each iteration, you should stat the proposals and the rejections.
\solution An $X$ indicates a rejection.\\
First iteration:
\begin{align*}
\text{Proposals}||&\text{Rejections}\\
j_1,a_1||&\\
j_2,a_1||&X\\
j_3,a_1||&X\\
j_4,a_2||&\\
j_5,a_3||&\\
\end{align*}
Second iteration:
\begin{align*}
\text{Proposals}||&\text{Rejections}\\
j_1,a_1||&\\
j_2,a_2||&\\
j_3,a_3||&X\\
j_4,a_2||&X\\
j_5,a_3||&\\
\end{align*}
\qed
\problem Answer questions about the graph $G$.
\begin{subproblems}
\item Find $k(G)$ and $k(G')$.
\item Give and example of a pair of vertices $x,y$ such that $k(x,y)>k(G)$.
\item Give an example of a disconnecting set of edges $F$ of $G$ that is minimal but not minimum.
\end{subproblems}
\subsol We have: $k(G)=k'(G)=2$.
\subsol Consider $(x,y)=(b_2,b_1)$. For this pair, $k(b_2,b_1)=4>2$.
\subsol Consider the disconnecting set $F=\{b_2a_z,b_2,a_3,b_2a_4,b_2a_5\}$.
\qed
\problem Assume $G$ is a connected graph on at least $3$ vertices. Prove that $G$ is 2-edge-connected if and only if every edge of $G$ lies on a cycle.
\solution Let $G$ be a connected graph on at least 3 vertices.\\
We proceed via contradiction. 
\begin{align*}
G\text{ is not 2-edge-connected}&\iff\exists\;S\subset E(G): G-S\text{ is disconnected and }|S|=1\\
&\iff G\text{ has a cut-edge }e\\
&\iff e\text{ is not on a cycle}.
\end{align*}
\qed
\problem Prove that if $G$ is $k$-connected and $G'$ is constructed from $G$ by adding a vertex $v$ with at least $k$ neighbors in $V(G)$, then $G'$ must also be $k$-connected.
\solution Let $S\subset V(G')$ be any separating set of $G'$. We consider 3 cases. If $v\in S$, then $G-(S+\{v\})$ is disconnected. Therefore, $|S|\ge k+1$. On the other hand, if $v\notin S$ but $N(v)\subseteq S$, then $G-S$ is disconnected. Hence, $|S|\ge k$. Lastly, if $v\notin N(S)$ and $v\notin S$, then $G-S$ is disconnected, so $|S|\ge k$. Since in every possible case, $|S|\ge k$, $G'$ is $k$-connected.
\qed
\problem Let $G$ be a $k$-connected graph. Let $A=\{ a_1,a_2,\dots,a_k\}$ and $B=\{b_1,b_2,\dots,b_k\}$ be disjoint subsets of $V(G)$ such that $|A|=|B|=k$. Prove that $G$ contains $k$ pairwise disjoint $A,B$-paths. (That is, $G$ contains $k$ paths each of which start with a vertex from the set $A$, end with a vertex from the set $B$ and share no vertices.)
\solution Create $G'$ by adding to $G$ two vertices, $v_A$ and $v_B$ and add $k$ edges to incident to both of them such that  $v_A\leftrightarrow a_i$ and $v_B\leftrightarrow b_i$ for every $1\le i\le k$. From the previous result, since $G$ is connected, $G'$ is $k$-connected as well. Hence, there exists $k$ pairwise disjoint $v_Av_B$-paths in $G'$ through $A$ and $B$. Deleting $v_A$ and $v_B$ leaves us with $G$, but there remains $k$ disjoint $A,B$-paths.
\qed
\end{problems}
\end{document}